%%%
%%%        Project: AraCom - Emotionale Robotik
%%%    Description: Introduction paper to emotional robotics
%%%        Version: 1.0
%%%         Author: Robert Jeutter <robert.jeutter@aracom.de>
%%%     Maintainer: Robert Jeutter <robert.jeutter@aracom.de>
%%%  Creation-Date: 24.05.2023
%%%      Copyright: (c) 2023 Robert Jeutter
%%%

\documentclass[aspectratio=169]{beamer}
%%%
%%%        Project: AraCom - LaTeX Template
%%%    Description: This is the basic LaTeX Template for all AraCom related presentations
%%%        Version: 1.0
%%%         Author: Robert Jeutter <robert.jeutter@aracom.de>
%%%     Maintainer: Robert Jeutter <robert.jeutter@aracom.de>
%%%  Creation-Date: 07.06.2023
%%%      Copyright: (c) 2023 Robert Jeutter
%%%      Images by AraCom IT Service
%%%

\usepackage{multirow}
\usepackage{subfigure}
\usepackage{etoolbox}
\usepackage{tikz}
\usepackage{listings}
\usepackage{graphicx}
\usepackage{xcolor}
\usepackage{amsfonts, amsmath, oldgerm, lmodern, animate}
\usepackage{verbatim}
\usepackage{bm}
\usepackage[T1]{fontenc}
\graphicspath{{./images/}}

\RequirePackage{arafont}

\definecolor{aracomblue}{RGB}{96, 167, 192}
\definecolor{aracomgrey}{rgb}{0.9, 0.9, 0.9}

\setbeamercolor{block title}{fg=white,bg=aracomblue}
\setbeamercolor{block body}{fg=white,bg=aracomblue}

\newcommand{\themecolor}[1]{
  \setbeamercolor{normal text}{fg=darkgray,bg=white}
  \setbeamercolor{structure}{fg=aracomblue}
  \setbeamercolor{block title}{fg=aracomblue,bg=aracomgrey}
  \setbeamercolor{block body}{fg=darkgray,bg=aracomgrey}
}

\themecolor{white}
\setbeamercolor{title}{fg=aracomblue}
\setbeamercolor{author}{fg=black}
\setbeamercolor{date}{fg=black}

\setbeamerfont{author}{size=\scriptsize}
\setbeamerfont{date}{size=\tiny}
\setbeamerfont{title}{series=\bfseries, size=\fontsize{36}{40}}
\setbeamerfont{subtitle}{series=\mdseries,size=\footnotesize}
\setbeamerfont{frametitle}{series=\bfseries}
\setbeamerfont{framesubtitle}{series=\mdseries}
\setbeamerfont{block title}{series=\centering, size=\small}
\setbeamerfont{block body}{size=\scriptsize}

% Code to get prettier boxes
\setbeamertemplate{blocks}[rounded, shadow=true]

% Bullets in several levels
\setbeamertemplate{itemize item}{\textbullet}
\setbeamertemplate{itemize subitem}{\textemdash}
\setbeamertemplate{itemize subsubitem}{\ensuremath{\circ}}

\newenvironment{colorblock}[3][white]{%
  \begingroup
  \setbeamercolor{block title}{fg=#1,bg=#2}
  \setbeamercolor{block body} {fg=#1,bg=#2}
  \begin{block}{#3}
    }{%
  \end{block}
  \endgroup
}

% Put the logo in each slide's down right area
\pgfdeclareimage[width=0.4\paperwidth]{araLogo}{./images/aracom_Logo-2023-white.png}
\renewcommand{\logo}{araLogo}
\setbeamertemplate{headline}{\vspace{1ex}\hspace{0.02\paperwidth} \LARGE{AraCom}}

% Define frame title and subtitle layout
\setbeamertemplate{frametitle}{
  \begin{beamercolorbox}[leftskip=2cm]{frametitle}%
    \usebeamerfont{frametitle}\insertframetitle\\
    \usebeamerfont{framesubtitle}\insertframesubtitle%
  \end{beamercolorbox}
}

% Define the title page
\setbeamertemplate{title page}{
  \vskip0pt plus 1filll
  \hspace{-12mm}% Pull back the box in an inelegant way - but it works!
  \begin{beamercolorbox}[wd=0.9\textwidth,sep=10pt,leftskip=8mm]{title}
    {\usebeamerfont{title}\inserttitle}

    %{\usebeamerfont{subtitle}\insertsubtitle}

    {\usebeamerfont{author}\usebeamercolor[fg]{author}\insertauthor}

    {\usebeamerfont{date}\usebeamercolor[fg]{date}\insertdate}
  \end{beamercolorbox}
  \setbeamertemplate{footline}{}
  \vspace{-.6ex}\hspace{0.57\paperwidth}\pgfuseimage{\logo}
}

\newcommand{\TikzSplitSlide}[1]{%
  \rule{0.4\paperwidth}{0pt}%
  \begin{tikzpicture}
    \clip (-0.1\paperwidth,-0.5\paperheight) --
    ( 0.5\paperwidth,-0.5\paperheight) --
    ( 0.5\paperwidth, 0.5\paperheight) --
    ( 0.1\paperwidth, 0.5\paperheight) -- cycle;
    \node at (0.2\paperwidth,0) {%
      \includegraphics[height=\paperheight]{#1}%
    };
  \end{tikzpicture}
}

\renewcommand{\maketitle}{
  \begingroup
  \setbeamertemplate{background}{
    \includegraphics[height=\paperheight]{./images/background.png}
  }
  \vspace{-5ex}\hspace{-5ex}\begin{frame}
    \Large\titlepage%
  \end{frame}
  \endgroup
}

\newenvironment{chapter}[3][]{% Args: image (optional), color, frame title
  \begingroup
  \themecolor{blue}
  \ifstrequal{#2}{aracomblue}{ % Use blue text, else white
    \setbeamercolor{frametitle}{fg=white}
    \setbeamercolor{normal text}{fg=white,bg=#2}
  }{
    \setbeamercolor{frametitle}{fg=aracomblue}
    \setbeamercolor{normal text}{fg=aracomblue,bg=#2}
  }
  \ifstrempty{#1}{}{\setbeamertemplate{background}{\TikzSplitSlide{#1}}}
  \setbeamertemplate{frametitle}{%
    \vspace*{8ex}
    \begin{beamercolorbox}[wd=0.45\textwidth]{frametitle}
      \usebeamerfont{frametitle}\insertframetitle\\
      \usebeamerfont{framesubtitle}\insertframesubtitle%
    \end{beamercolorbox}
  }
  \begin{frame}{#3}
    \hspace*{0.05\textwidth}%
    \minipage{0.35\textwidth}%
    \usebeamercolor[fg]{normal text}%
    }{%
    \endminipage%
  \end{frame}
  \endgroup
}

\newenvironment{sidepic}[2]{% Args: image, frame title
  \begingroup
  \setbeamertemplate{background}{%
    \hspace*{0.6\paperwidth}%
    \includegraphics[height=\paperheight]{#1}%
  }
  \setbeamertemplate{frametitle}{% Same as normal, but with right skip
    \vspace*{-3.5ex}
    \begin{beamercolorbox}[leftskip=2cm,rightskip=0.4\textwidth]{frametitle}%
      \usebeamerfont{frametitle}\insertframetitle\\
      \usebeamerfont{framesubtitle}\insertframesubtitle%
    \end{beamercolorbox}
  }
  \begin{frame}{#2}
    \minipage{0.6\textwidth}%
    }{%
    \endminipage%
  \end{frame}
  \endgroup
}

\newcommand{\strtoc}{Table of Contents}
\newcommand{\strsubsec}{Section \thesection.\thesubsection}

% TYPESETTING ELEMENTS

% style of section presented in the table of contents
\setbeamertemplate{section in toc}{$\blacktriangleright$~\inserttocsection}

% style of subsection presented in the table of contents
\setbeamertemplate{subsection in toc}{}
\setbeamertemplate{subsection in toc}{\textcolor{white}\footnotesize\hspace{1.2 em}$\blacktriangleright$~\inserttocsubsection}

% automate subtitle of each frame
\makeatletter
%\pretocmd\beamer@checkframetitle{\framesubtitle{\thesection\, \secname}}
\makeatother

% avoid numbering of frames that are breaked into multiply slides
\setbeamertemplate{frametitle continuation}{}

% at the begining of section, add table of contents with the current section highlighted
\AtBeginSection[]
{
  \begingroup
  \themecolor{blue}
  \begin{chapter}[./images/titleimage.png]{white}{Übersicht}
     \tableofcontents[currentsection]
  \end{chapter}
  \endgroup
}

% at the beginning of subsection, add subsection title page
\AtBeginSubsection[]
{
  \begin{frame}{\,}{\thesection\, \secname}
    \fontfamily{ptm}\selectfont
    \centering\textsl{\textbf{\textcolor{aracomblue}{
          \large Section \thesection.\thesubsection%
          \vskip15pt
          \LARGE \subsecname%
        }}}
  \end{frame}
}

% code bolck setting
\definecolor{codegreen}{RGB}{101,218,120}
\definecolor{codegray}{rgb}{0.5,0.5,0.5}
\definecolor{codepurple}{rgb}{0.58,0,0.82}
\definecolor{backcolour}{rgb}{0.95,0.95,0.92}

\lstdefinestyle{mystyle}{
  % backgroundcolor=\color{backcolour},
  commentstyle=\color{aracomblue},
  keywordstyle=\color{magenta},
  numberstyle=\tiny\color{codegray},
  stringstyle=\color{codepurple},
  basicstyle=\ttfamily\scriptsize,
  breakatwhitespace=false,
  breaklines=true,
  captionpos=b,
  keepspaces=true,
  numbers=left,
  numbersep=5pt,
  showspaces=false,
  showstringspaces=false,
  showtabs=false,
  tabsize=4,
  xleftmargin=10pt,
  xrightmargin=10pt,
}

\lstset{style=mystyle}

% NEW COMMANDS

% set colored hyperlinks command
\newcommand{\hrefcol}[2]{\textcolor{aracomgrey}{\href{#1}{#2}}}
\newcommand{\hlinkcol}[1]{\hrefcol{#1}{#1}}


% centering paragraph statement
\newcommand{\centerstate}[1]{
  \centering
  \begin{columns}
    \begin{column}{0.8\textwidth}
      #1
    \end{column}
  \end{columns}
}

% colored textbf
\newcommand{\atextbf}[1]{\textbf{\textcolor{aracomblue}{#1}}}
\newcommand{\atextsl}[1]{\textsl{\textcolor{aracomblue}{#1}}}
\newcommand{\aemph}[1]{\emph{\textcolor{aracomblue}{#1}}}

% about page
\newcommand{\aboutpage}[2]{
  \begingroup
  \themecolor{blue}
  \begin{frame}[c]{#1}{\,}
    \centering
    \begin{minipage}{\textwidth}
      \usebeamercolor[fg]{normal text}
      \centering
      \Large \textsl{\normalsize #2}
    \end{minipage}
  \end{frame}
  \endgroup
}

% bibliography page
\newcommand{\bibliographpage}{
  \section{References}

  \begingroup
  \themecolor{blue}
  \begin{frame}[allowframebreaks]{References}{\,}
    \tiny
    \printbibliography[heading=none]
  \end{frame}
  \endgroup
}

\title{Emotionale Robotik}
\subtitle{Eine Einführung in die soziale Robotik}
\author{Robert Jeutter \& Dennis Eisermann}
\date{05. Oktober 2023}

\begin{document}

\maketitle

\section{Über uns}
\begin{frame}{über uns}
  \begin{columns}
    \begin{column}{0.45\textwidth}
      \begin{figure}[h]
        \centering
        \includegraphics[width=.7\linewidth]{images/robert.png}\\
        \textbf{Robert Jeutter}, 25,\\
        Softwareentwickler,\\
        \scriptsize{Schwerpunkte maschinelles Lernen und Mensch-Maschine-Interaktion}
      \end{figure}
    \end{column}
    \begin{column}{0.45\textwidth}
      \includegraphics[width=.7\linewidth]{images/dennis.png}\\
      \textbf{Dennis Eisermann}, 24,\\
      Softwareentwickler \& Student,\\
      \scriptsize{Schwerpunkte maschinelles Lernen und IT Security}
    \end{column}
  \end{columns}
\end{frame}
\begin{frame}{die AraCom IT Service GmbH}
  \includegraphics[width=.6\linewidth]{images/AraCom-Logo-black.png}
  IT Service GmbH\\
  der Spezialist für maßgeschneiderte Software\\
  Hauptsitz in Gersthofen, weitere in München, Stuttgart und Bamberg\\
  bietet Ausbildungsplätze \& spannende Projekte
\end{frame}
\begin{frame}{Sponsor des RoboCup}
  \includegraphics[height=30px]{images/AraCom-Logo-black.png} sponsort
  \includegraphics[height=35px]{images/RCJ_Logo.png}
  \begin{columns}
    \begin{column}{0.45\textwidth}
      \begin{figure}[h]
        \centering
        \includegraphics[width=\linewidth]{images/RoboCupFinale2023.jpg}
        %~ \cite{robocup} RoboCup Soccer Roboter
      \end{figure}
    \end{column}
    \begin{column}{0.45\textwidth}
      \begin{itemize}
        \item Robotik Wettbewerb
        \item Soccer, Rescue \& OnStage
        \item selbst gebaut \& programmiert
        \item jährliche Austragung weltweit
        \item bis 2050 gegen Fifa Weltmeister
      \end{itemize}
    \end{column}
  \end{columns}
\end{frame}

\begin{frame}{Definition von Emotionen}
  \begin{itemize}
      \item Emotionen sind komplexe psychologische Zustände.
      \item Sie beinhalten drei unterschiedliche Komponenten:
      \begin{itemize}
          \item Ein subjektives Erleben.
          \item Eine physiologische Reaktion.
          \item Eine verhaltensbezogene oder expressive Antwort.
      \end{itemize}
      \item Emotionen sind vielschichtige Antworten auf Reize.
      \item Sie können durch unsere Umgebung oder Erinnerungen ausgelöst werden.
      \item Typischerweise von kurzer Dauer.
      \item Kann aus verbalen, physiologischen, verhaltensbezogenen und neuronalen Mechanismen bestehen.
  \end{itemize}
\end{frame}

\begin{frame}{Verhaltensbezogene oder expressive Antwort von Emotionen}
  \begin{itemize}
      \item Bezieht sich auf den sichtbaren Ausdruck von Emotionen.
      \item Gibt Hinweise über den emotionalen Zustand.
      \item Kommuniziert Absichten und Gefühle an andere.
      \item Beeinflusst soziale Interaktionen und Beziehungen.
      \item Kann automatisch oder bewusst gesteuert werden.
  \end{itemize}
\end{frame}

\begin{frame}{Emotionen auszudrücken}
  \begin{itemize}
      \item Gesichtsausdrücke, z.B. Lächeln oder Stirnrunzeln.
      \item Gesten, z.B. Klatschen oder Zeigen.
      \item Körpersprache, z.B. Körperhaltung oder Nähe.
      \item Stimmklang, z.B. Schreien oder Flüstern.
  \end{itemize}
\end{frame}

\begin{frame}{Definition - Soziale und Emotionale Intelligenz}
  \textbf{Definition sozialer Intelligenz:}
  \begin{quote}
      Verstehen und Navigieren in sozialen Umgebungen und Interaktionen.
  \end{quote}
  
  \textbf{Definition emotionaler Intelligenz:}
  \begin{quote}
      Erkennen, Verstehen und Steuern der eigenen Emotionen sowie der Emotionen anderer.
  \end{quote}
\end{frame}

\begin{frame}{Verschmelzende Domänen bei emotionaler Robotik}
  \begin{itemize}
      \item Robotik
      \item Psychologie
      \item Künstlicher Intelligenz
  \end{itemize}
\end{frame}

\begin{frame}{Aufstieg der Emotionalen Robotik}
  \textbf{Ursprünge der emotionalen Robotik:}
  \begin{itemize}
      \item Erkennen menschlicher Emotionen
      \item Reagieren und Interagieren auf Basis dieser Emotionen
  \end{itemize}
  
  \textbf{Aktueller Stand:}
  \begin{itemize}
      \item Emotionserkennung
      \item Feedback-Mechanismen
      \item Anwendungen in realen Szenarien
  \end{itemize}
\end{frame}

\begin{frame}{Ziele und Vorteile der Emotionalen Robotik}
  \textbf{Hauptziele:}
  \begin{itemize}
      \item Vertiefung der Mensch-Maschine-Interaktionen
      \item Personalisierte Benutzererfahrungen
  \end{itemize}
  
  \textbf{Wesentliche Vorteile:}
  \begin{itemize}
      \item Emotionaler Support
      \item Motivation und Ansporn
      \item Steigerung der Geselligkeit
      \item Verbesserung der Lebensqualität
  \end{itemize}
\end{frame}

\begin{frame}{Risiken und Bedenken}
  \begin{itemize}
      \item Datenschutz und -sicherheit
      \item Emotionale Manipulation durch Maschinen
      \item Potenzielle Voreingenommenheiten bei der Emotionserkennung
  \end{itemize}
\end{frame}

\begin{frame}{Anwendungen und Einsatzmöglichkeiten}
  \textbf{Aktuelle Einsatzbereiche:}
  \begin{itemize}
      \item Therapeutische Umgebungen
      \item Kundenservice
  \end{itemize}

  \textbf{Potenzielle Bereiche:}
  \begin{itemize}
      \item Pflege von älteren Menschen
      \item Unterstützung der psychischen Gesundheit
      \item Personalisiertes Tutoring
  \end{itemize}
\end{frame}

\begin{frame}{Einführung von Emotionaler Robotik im Betrieb}
  \textbf{Technologische Anforderungen:}
  \begin{itemize}
      \item Robotik: Sensoren, Prozessoren, Aktoren 
      \item ML-Modelle: Umgebungserkennung, Schlussfolgern
  \end{itemize}
  \textbf{Training und Datenerfassung:}
  \begin{itemize}
      \item Erfassung emotionaler Daten
      \item Verfeinerung von Maschinenantworten
      \item Feedback-Schleifen zwischen Mensch und Maschine
  \end{itemize}

  \textbf{Ethische Anforderungen}
  \begin{itemize}
      \item Richtlinien und Vorsichtsmaßnahmen für Unternehmen
      \item Akzeptanz von maschinellen Lösungen schaffen
      \item Differenz zwischen Mensch und Maschine
  \end{itemize}
\end{frame}

\begin{frame}{Zukünftige Horizonte}
  \begin{itemize}
      \item Fortgeschrittene Modelle, die kulturelle Unterschiede berücksichtigen
      \item Roboter, die künstliche Empathie zeigen
      \item Generierung und Ausdruck einzigartiger Roboter-"Emotionen"
  \end{itemize}
\end{frame}

\begin{frame}{Anthropomorphismus und Vertrauen}
  \begin{itemize}
      \item Anthropomorphismus: Zuweisung menschlicher Eigenschaften
  \end{itemize}
  \textbf{Die Implikationen für Roboter:}
  \begin{itemize}
      \item Erhöhtes Vertrauen
      \item Das "Uncanny Valley" (unheimliche Tal)
  \end{itemize}
  \textbf{Herausforderungen:}
  \begin{itemize}
      \item Ängste im Zusammenhang mit Technologie ansprechen
      \item Akzeptanz fördern
  \end{itemize}
\end{frame}

\begin{frame}{Das "Uncanny Valley" (unheimliche Tal)}
  \begin{itemize}
      \item Konzept in Robotik und Animation.
      \item Unbehagen durch fast-menschliche Darstellung.
      \item 1970 von Masahiro Mori geprägt.
      \item Höchstes Unwohlsein vor perfekter Menschlichkeit.
      \item Beeinflusst Technologieakzeptanz.
  \end{itemize}
\end{frame}

\begin{frame}{Fragen \& Antworten}
  Öffnung für Fragen und Förderung eines beidseitigen Dialogs.
\end{frame}

\begin{frame}{Schlussfolgerung und Schlüsselerkenntnisse}
  \begin{itemize}
      \item Zusammenfassung des Potenzials und der Herausforderungen der emotionalen Robotik
      \item Ermutigung zur weiteren Erkundung und zum Verständnis in diesem Bereich
      \item Einladung zu Zusammenarbeiten und Diskussionen nach der Präsentation
  \end{itemize}
\end{frame}

\begin{frame}{Danksagungen und Referenzen}
  \textbf{Danksagungen:}
  \begin{itemize}
      \item Aracom IT-Services GmbH für die Unterstützung bei der Erstellung dieser Präsentation
      \item Hackerkiste für die Organisation dieses tollen Events
  \end{itemize}
  
  \textbf{Referenzen:}
  \begin{enumerate}
      \item Goleman, D. (2006). \textit{Social Intelligence: The New Science of Human Relationships}. Bantam Books.
      \item Goleman, D. (1995). \textit{Emotional Intelligence}. Bantam Books. 
      \item Siciliano, B., & Khatib, O. (Eds.). (2016). \textit{Springer Handbook of Robotics}. Springer. [For general knowledge on Robotics]
      \item Goodfellow, I., Bengio, Y., & Courville, A. (2016). \textit{Deep learning}. MIT press. [For Machine Learning and its convergence with AI]
      \item Russell, S. J., & Norvig, P. (2020). \textit{Artificial Intelligence: A Modern Approach}. Malaysia; Pearson Education Limited. [For Artificial Intelligence]
      \item Breazeal, C. (2003). Emotion and sociable humanoid robots. \textit{International Journal of Human-Computer Studies}, 59(1-2), 119-155. [For emotional robotics]
      \item Whitby, B. (2012). Do you want a robot lover?. \textit{The Machine Question}, 3-16. [For emotional manipulation by machines]
      \item Zadeh, L. A. (1973). Outline of a new approach to the analysis of complex systems and decision processes. \textit{IEEE Transactions on Systems, Man, and Cybernetics}, SMC-3(1), 28-44.
      \item Weizenbaum, J. (1966). ELIZA—a computer program for the study of natural language communication between man and machine. \textit{Communications of the ACM}, 9(1), 36-45.
      \item Takanishi, A., & Ishiguro, H. (2018). Pioneers in robotics: Hiroshi Ishiguro—From android science to interactive robots. \textit{IEEE Robotics \& Automation Magazine}, 25(3), 10-14.
      \item Wada, K., Shibata, T., Musha, T., & Kimura, S. (2008). Robot therapy for elders affected by dementia. \textit{IEEE Engineering in Medicine and Biology Magazine}, 27(4), 53-60.
      \item Kanade, T., Cohn, J. F., & Tian, Y. (2000). Comprehensive database for facial expression analysis. In \textit{Fourth IEEE International Conference on Automatic Face and Gesture Recognition} (pp. 46-53).
      \item Zeng, Z., Pantic, M., Roisman, G. I., & Huang, T. S. (2009). A survey of affect recognition methods: Audio, visual, and spontaneous expressions. \textit{IEEE transactions on pattern analysis and machine intelligence}, 31(1), 39-58.
      \item Epley, N., Waytz, A., & Cacioppo, J. T. (2007). On seeing human: A three-factor theory of anthropomorphism. \textit{Psychological Review}, 114(4), 864.
      \item Mori, M., MacDorman, K. F., & Kageki, N. (2012). The uncanny valley [From the field]. \textit{IEEE Robotics & Automation Magazine}, 19(2), 98-100.
      \item Riek, L. D., Rabinowitch, T. C., Chakrabarti, B., & Robinson, P. (2009). How anthropomorphism affects empathy toward robots. In \textit{4th ACM/IEEE international conference on human-robot interaction (HRI)} (pp. 245-246).
      \item Mori, M., MacDorman, K. F., & Kageki, N. (2012). Das unheimliche Tal [From the field]. \textit{IEEE Robotics \& Automation Magazine}, 19(2), 98-100.
      \item Bartneck, C., Kanda, T., Ishiguro, H., & Hagita, N. (2007). Is the Uncanny Valley an Uncanny Cliff? \textit{Proceedings of the 16th IEEE}, 368-373.
  \end{enumerate}
\end{frame}

\begin{frame}[c]{}
  \centering
  \begin{minipage}{\textwidth}
    \usebeamercolor[fg]{normal text}
    \centering
    \Large \atextsl{Alle Folien als CC-BY-SA verfügbar auf}\\
    \href{https://github.com/wieerwill/emotional-robotics}{GitHub.com/WieErWill/emotional-robotics}\\
    \vspace{.4cm}
    \includegraphics[width=.4\linewidth]{images/qrcode-repo.png}
  \end{minipage}
\end{frame}

\begin{frame}[c]{}
  \centering
  \begin{minipage}{\textwidth}
    \usebeamercolor[fg]{normal text}
    \centering
    \Huge \[\mathcal Q \& \mathcal A\]
    \Large \atextsl{
      Vielen Dank für eure Aufmerksamkeit! \\
      Feedback und Diskussionen sind sehr willkommen \\
    }
  \end{minipage}
\end{frame}

\section{Quellen \& Literatur}
\begin{frame}[allowframebreaks]{Quellen \& Literatur}
  \scriptsize
  \bibliography{../bibliography}
  \bibliographystyle{unsrt}
\end{frame}

\end{document}