%%%
%%%        Project: AraCom - Emotionale Robotik
%%%    Description: Introduction paper to emotional robotics
%%%        Version: 1.0
%%%         Author: Robert Jeutter <robert.jeutter@aracom.de>
%%%     Maintainer: Robert Jeutter <robert.jeutter@aracom.de>
%%%  Creation-Date: 24.05.2023
%%%      Copyright: (c) 2023 Robert Jeutter
%%%

\documentclass[aspectratio=169]{beamer}
%%%
%%%        Project: AraCom - LaTeX Template
%%%    Description: This is the basic LaTeX Template for all AraCom related presentations
%%%        Version: 1.0
%%%         Author: Robert Jeutter <robert.jeutter@aracom.de>
%%%     Maintainer: Robert Jeutter <robert.jeutter@aracom.de>
%%%  Creation-Date: 07.06.2023
%%%      Copyright: (c) 2023 Robert Jeutter
%%%      Images by AraCom IT Service
%%%

\usepackage{multirow}
\usepackage{subfigure}
\usepackage{etoolbox}
\usepackage{tikz}
\usepackage{listings}
\usepackage{graphicx}
\usepackage{xcolor}
\usepackage{amsfonts, amsmath, oldgerm, lmodern, animate}
\usepackage{verbatim}
\usepackage{bm}
\usefonttheme{serif}

\graphicspath{{./images/}}

\definecolor{aracomblue}{RGB}{38, 113, 142}
\definecolor{aracomred}{RGB}{143, 53, 70}
\definecolor{aracomgrey}{rgb}{0.9, 0.9, 0.9}

\setbeamercolor{block title}{fg=white,bg=aracomblue}
\setbeamercolor{block body}{fg=white,bg=aracomblue}

\newcommand{\themecolor}[1]{
		\setbeamercolor{normal text}{fg=darkgray,bg=white}
		\setbeamercolor{structure}{fg=aracomblue}
		\setbeamercolor{block title}{fg=aracomblue,bg=aracomgrey}
		\setbeamercolor{block body}{fg=darkgray,bg=aracomgrey}
}

\themecolor{white}
\setbeamercolor{title}{fg=aracomblue}
\setbeamercolor{alerted text}{fg=aracomred}
\setbeamercolor{author}{fg=black}
\setbeamercolor{date}{fg=black}

\setbeamerfont{author}{size=\scriptsize}
\setbeamerfont{date}{size=\tiny}
\setbeamerfont{title}{series=\bfseries}
\setbeamerfont{subtitle}{series=\mdseries,size=\footnotesize}
\setbeamerfont{frametitle}{series=\bfseries}
\setbeamerfont{framesubtitle}{series=\mdseries}
\setbeamerfont{block title}{series=\centering, size=\small}
\setbeamerfont{block body}{size=\scriptsize}

% Code to get prettier boxes
\setbeamertemplate{blocks}[rounded, shadow=true]

% Bullets in several levels
\setbeamertemplate{itemize item}{\textbullet}
\setbeamertemplate{itemize subitem}{\textemdash}
\setbeamertemplate{itemize subsubitem}{\ensuremath{\circ}}

\newenvironment{colorblock}[3][white]{%
	\begingroup
	\setbeamercolor{block title}{fg=#1,bg=#2}
	\setbeamercolor{block body} {fg=#1,bg=#2}
	\begin{block}{#3}
	}{%
	\end{block}
	\endgroup
}

% Put the logo in each slide's top left area
\pgfdeclareimage[width=0.1\paperwidth]{bluelogo}{./images/AraCom-Logo-black.png}
\renewcommand{\logo}{bluelogo}
\setbeamertemplate{headline}{\vspace{1ex}\hspace{0.03\textwidth}\pgfuseimage{\logo}}

% Define frame title and subtitle layout
\setbeamertemplate{frametitle}{
  \begin{beamercolorbox}[leftskip=2cm]{frametitle}%
    \usebeamerfont{frametitle}\insertframetitle\\
    \usebeamerfont{framesubtitle}\insertframesubtitle%
  \end{beamercolorbox}
}

% Define the title page
\setbeamertemplate{title page}{
  \vskip0pt plus 1filll
  \hspace{-12mm}% Pull back the box in an inelegant way - but it works!
  \begin{beamercolorbox}[wd=0.9\textwidth,sep=10pt,leftskip=8mm]{title}
    {\usebeamerfont{title}\inserttitle}

    {\usebeamerfont{subtitle}\insertsubtitle}

    {\usebeamerfont{author}\usebeamercolor[fg]{author}\insertauthor}

    {\usebeamerfont{date}\usebeamercolor[fg]{date}\insertdate}
  \end{beamercolorbox}
}

\newcommand{\TikzSplitSlide}[1]{%
  \rule{0.4\paperwidth}{0pt}%
  \begin{tikzpicture}
    \clip (-0.1\paperwidth,-0.5\paperheight) --
          ( 0.5\paperwidth,-0.5\paperheight) --
          ( 0.5\paperwidth, 0.5\paperheight) --
          ( 0.1\paperwidth, 0.5\paperheight) -- cycle;
    \node at (0.2\paperwidth,0) {%
      \includegraphics[height=\paperheight]{#1}%
    };
  \end{tikzpicture}
}

\renewcommand{\maketitle}{
\begingroup
  \setbeamertemplate{background}{
      \includegraphics[height=\paperheight]{./images/background.png}
  }
  \vspace{-5ex}\hspace{-5ex}\begin{frame}
      \Large\titlepage%
  \end{frame}
\endgroup
}

\newenvironment{chapter}[3][]{% Args: image (optional), color, frame title
  \begingroup
  \themecolor{blue}
  \ifstrequal{#2}{aracomlightgreen}{ % Use blue text on light green, else white
    \setbeamercolor{frametitle}{fg=aracomblue}
    \setbeamercolor{normal text}{fg=aracomblue,bg=#2}
  }{
    \setbeamercolor{frametitle}{fg=white}
    \setbeamercolor{normal text}{fg=white,bg=#2}
  }
  \ifstrempty{#1}{}{\setbeamertemplate{background}{\TikzSplitSlide{#1}}}
  \setbeamertemplate{frametitle}{%
    \vspace*{8ex}
    \begin{beamercolorbox}[wd=0.45\textwidth]{frametitle}
      \usebeamerfont{frametitle}\insertframetitle\\
      \usebeamerfont{framesubtitle}\insertframesubtitle%
    \end{beamercolorbox}
  }
  \begin{frame}{#3}
  \hspace*{0.05\textwidth}%
  \minipage{0.35\textwidth}%
  \usebeamercolor[fg]{normal text}%
}{%
  \endminipage%
  \end{frame}
  \endgroup
}

\newenvironment{sidepic}[2]{% Args: image, frame title
  \begingroup
  \setbeamertemplate{background}{%
  \hspace*{0.6\paperwidth}%
  \includegraphics[height=\paperheight]{#1}%
  }
  \setbeamertemplate{frametitle}{% Same as normal, but with right skip
    \vspace*{-3.5ex}
    \begin{beamercolorbox}[leftskip=2cm,rightskip=0.4\textwidth]{frametitle}%
      \usebeamerfont{frametitle}\insertframetitle\\
      \usebeamerfont{framesubtitle}\insertframesubtitle%
    \end{beamercolorbox}
  }
  \begin{frame}{#2}
  \minipage{0.6\textwidth}%
}{%
  \endminipage%
  \end{frame}
  \endgroup
}

\newcommand{\strtoc}{Table of Contents}
\newcommand{\strsubsec}{Section \thesection.\thesubsection}

% TYPESETTING ELEMENTS

% style of section presented in the table of contents
\setbeamertemplate{section in toc}{$\blacktriangleright$~\inserttocsection}

% style of subsection presented in the table of contents
\setbeamertemplate{subsection in toc}{}
% \setbeamertemplate{subsection in toc}{\footnotesize\hspace{1.2 em}$\blacktriangleright$~\inserttocsubsection}

% automate subtitle of each frame
\makeatletter
    \pretocmd\beamer@checkframetitle{\framesubtitle{\thesection\, \secname}}
\makeatother

% avoid numbering of frames that are breaked into multiply slides
\setbeamertemplate{frametitle continuation}{}

% at the begining of section, add table of contents with the current section highlighted
\AtBeginSection[]
{
    \begingroup
    \themecolor{blue}
    \begin{frame}{Table of Contents}
        \tableofcontents[currentsection]
    \end{frame}
    \endgroup
}

% at the beginning of subsection, add subsection title page
\AtBeginSubsection[]
{
    \begin{frame}{\,}{\thesection\, \secname}
        \fontfamily{ptm}\selectfont
        \centering\textsl{\textbf{\textcolor{aracomblue}{
            \large Section \thesection.\thesubsection%
            \vskip15pt
            \LARGE \subsecname%
        }}}
    \end{frame}
}

% code bolck setting
\definecolor{codegreen}{RGB}{101,218,120}
\definecolor{codegray}{rgb}{0.5,0.5,0.5}
\definecolor{codepurple}{rgb}{0.58,0,0.82}
\definecolor{backcolour}{rgb}{0.95,0.95,0.92}

\lstdefinestyle{mystyle}{
    % backgroundcolor=\color{backcolour},
    commentstyle=\color{aracomblue},
    keywordstyle=\color{magenta},
    numberstyle=\tiny\color{codegray},
    stringstyle=\color{codepurple},
    basicstyle=\ttfamily\scriptsize,
    breakatwhitespace=false,
    breaklines=true,
    captionpos=b,
    keepspaces=true,
    numbers=left,
    numbersep=5pt,
    showspaces=false,
    showstringspaces=false,
    showtabs=false,
    tabsize=4,
    xleftmargin=10pt,
    xrightmargin=10pt,
}

\lstset{style=mystyle}

% NEW COMMANDS

% set colored hyperlinks command
\newcommand{\hrefcol}[2]{\textcolor{aracomblue}{\href{#1}{#2}}}
\newcommand{\hlinkcol}[1]{\hrefcol{#1}{#1}}


% centering paragraph statement
\newcommand{\centerstate}[1]{
    \centering
    \begin{columns}
        \begin{column}{0.8\textwidth}
            #1
        \end{column}
    \end{columns}
}

% colored textbf
\newcommand{\atextbf}[1]{\textbf{\textcolor{aracomblue}{#1}}}
\newcommand{\atextsl}[1]{\textsl{\textcolor{aracomblue}{#1}}}
\newcommand{\aemph}[1]{\emph{\textcolor{aracomblue}{#1}}}

% about page
\newcommand{\aboutpage}[2]{
    \begingroup
    \themecolor{blue}
    \begin{frame}[c]{#1}{\,}
        \centering
        \begin{minipage}{\textwidth}
            \usebeamercolor[fg]{normal text}
            \centering
            \Large \textsl{\normalsize #2}
        \end{minipage}
    \end{frame}
    \endgroup
}

% bibliography page
\newcommand{\bibliographpage}{
    \section{References}

    \begingroup
    \themecolor{blue}
    \begin{frame}[allowframebreaks]{References}{\,}
        \tiny
        \printbibliography[heading=none]
    \end{frame}
\endgroup
}

\title{Emotionale Robotik}
\subtitle{Eine Einführung in die soziale Robotik}
\author{Robert Jeutter}
\date{05. Oktober 2023}

\begin{document}

\maketitle

\section{Über uns}
\begin{frame}{über uns}
  \begin{columns}
    \begin{column}{0.45\textwidth}
      \begin{figure}[h]
        \centering
        \includegraphics[width=.8\linewidth]{images/robert.jpg}\\
        \textbf{Robert Jeutter}, 25,\\
        Softwareentwickler,\\
        \scriptsize{Schwerpunkte maschinelles Lernen und Mensch-Maschine-Interaktion}
      \end{figure}
    \end{column}
    \begin{column}{0.45\textwidth}
      \includegraphics[width=.6\linewidth]{images/dennis.png}\\
      \textbf{Dennis Eisermann}, 24,\\
      Softwareentwickler \& Student,\\
      \scriptsize{Schwerpunkte maschinelles Lernen und IT Security}
    \end{column}
  \end{columns}
\end{frame}
\begin{frame}{die AraCom IT Service GmbH}
  \includegraphics[width=.6\linewidth]{images/AraCom-Logo-black.png}
  IT Service GmbH\\
  der Spezialist für maßgeschneiderte Software\\
  Hauptsitz in Gersthofen, weitere in München, Stuttgart und Bamberg\\
  bietet Ausbildungsplätze \& spannende Projekte
\end{frame}
\begin{frame}{Sponsor des RoboCup}
  \includegraphics[height=30px]{images/AraCom-Logo-black.png} sponsort
  \includegraphics[height=35px]{images/RCJ_Logo.png}
  \begin{columns}
    \begin{column}{0.45\textwidth}
      \begin{figure}[h]
        \centering
        \includegraphics[width=\linewidth]{images/RoboCupFinale2023.jpg}
        %~ \cite{robocup} RoboCup Soccer Roboter
      \end{figure}
    \end{column}
    \begin{column}{0.45\textwidth}
      \begin{itemize}
        \item Robotik Wettbewerb
        \item Soccer, Rescue \& OnStage
        \item selbst gebaut \& programmiert
        \item jährliche Austragung weltweit
        \item bis 2050 gegen Fifa Weltmeister
      \end{itemize}
    \end{column}
  \end{columns}
\end{frame}

\section{Soziale Robotik -- Emotionale Robotik}
\begin{frame}{Was ist soziale Robotik}
\end{frame}
\begin{frame}{Nutzen der sozialen Robotik}

  Emotionaler Support
  %Roboter können Menschen emotional unterstützen, indem sie empathisch reagieren, aufmerksam zuhören und tröstende Gesten oder Worte anbieten.

  Gesellschaft
  %Insbesondere für ältere Menschen, die möglicherweise isoliert oder einsam sind, kann ein sozial interaktiver Roboter Gesellschaft bieten und dazu beitragen, soziale Interaktionen aufrechtzuerhalten.

  Stressabbau
  %Durch verschiedene Interaktionsmöglichkeiten wie Entspannungsübungen, Atemübungen oder beruhigende Musik können Roboter dazu beitragen, Stress abzubauen und eine entspannte Atmosphäre zu schaffen.

  Therapeutische Unterstützung
  %In therapeutischen Kontexten können Roboter zur Unterstützung von Menschen mit bestimmten Bedürfnissen eingesetzt werden, z. B. bei der Behandlung von Autismus oder bei der Rehabilitation nach Verletzungen.

\end{frame}
\begin{frame}{Nutzen der sozialen Robotik}
  Unterstützung bei alltäglichen Aufgaben
  % Ein Roboter, der sozial interagieren kann, kann Menschen im Haushalt unterstützen, beispielsweise indem er Erinnerungen an Termine oder Medikamente gibt, beim Einkaufen hilft oder bei der Organisation des Tagesablaufs unterstützt.

  Bildung und Information
  % Durch soziale Interaktion können Roboter Wissen und Informationen vermitteln. Sie können als Lehrer oder Sprachassistenten fungieren und den Nutzern helfen, neue Fähigkeiten zu erlernen oder auf Fragen zu antworten.

  Motivation und Ansporn
  %Ein Roboter, der positive Verstärkung gibt und ermutigende Worte spricht, kann Menschen dabei unterstützen, ihre Ziele zu erreichen und motiviert zu bleiben.

  Unterhaltung
  % Durch soziale Interaktion und interaktive Spiele können Roboter eine unterhaltsame Quelle der Freizeitgestaltung sein und Menschen dabei helfen, sich zu entspannen und Spaß zu haben.

  Verbesserung der Lebensqualität
  %Insgesamt kann die soziale Interaktion mit Robotern die Lebensqualität der Menschen verbessern, indem sie ihnen Unterstützung, Komfort und eine neue Art der Interaktion bietet.
\end{frame}
\begin{frame}{Bereiche der sozialen Robotik}

\end{frame}
\begin{frame}{Beispiel anhand des AraCom Roboters}

\end{frame}
\begin{frame}{Ausblick OpenSpace \& weitere Entwicklung des Roboters}

\end{frame}

\begin{frame}[c]{}
  \centering
  \begin{minipage}{\textwidth}
    \usebeamercolor[fg]{normal text}
    \centering
    \Large \atextsl{Alle Folien als CC-BY-SA verfügbar auf}\\
    \href{https://github.com/wieerwill/emotional-robotics}{GitHub.com/WieErWill/emotional-robotics}\\
    \vspace{.4cm}
    \includegraphics[width=.4\linewidth]{images/qrcode-repo.png}
  \end{minipage}
\end{frame}

\begin{frame}[c]{}
  \centering
  \begin{minipage}{\textwidth}
    \usebeamercolor[fg]{normal text}
    \centering
    \Huge \[\mathcal Q \& \mathcal A\]
    \Large \atextsl{
      Vielen Dank für eure Aufmerksamkeit! \\
      Feedback und Diskussionen sind sehr willkommen \\
    }
  \end{minipage}
\end{frame}

\section{Quellen \& Literatur}
\begin{frame}[allowframebreaks]{Quellen \& Literatur}
  \scriptsize
  \bibliography{../bibliography}
  \bibliographystyle{unsrt}
\end{frame}

\end{document}